\documentclass[11pt,fleqn]{article}
\usepackage{amsmath,amssymb}
\usepackage[utf8]{inputenc} \usepackage[letterpaper, total={6.5in, 8.5in}]{geometry}
\usepackage{multicol}
\setlength\columnsep{30pt}
\usepackage{enumerate}
\usepackage{listings}
\lstset{breaklines,basicstyle=\ttfamily,columns=fullflexible,keepspaces=true,breakindent=0pt,belowskip=-8pt}

\begin{document}
\title{Proyecto de Búsqueda}
\author{
	Cheung, Derek \\
	González Rico, Diana Virginia \\
	Neri González, José Francisco
}
\date{23 noviembre 2016}
\maketitle

\section{Módulo de Diagnóstico}

\paragraph{Descripción} El módulo de diagnóstico toma todas las creencias sobre el escenario y lo comparan contra las observaciónes. En caso de que haya contradiciones, genera un diagnóstico posible para explicar posibles razones.

\subsection*{Nombre}
	\begin{verbatim}
		diagnosis(KB, Creencias_New, Acciones)
	\end{verbatim}

\subsection*{Argumentos}
	\textit{KB}: (Input) Base de conocimiento a consultar \\
	\textit{Creencias\_New}: Lista de creencias representado en forma de clave y valor \\
	\textit{Acciones}: Lista de acciones que hizo el asistente

\paragraph{Búsqueda} El algoritmo elige un diagnóstico que es lo más parecido al esperadosegún el número de objectos en cada estante. Es decir, prefiere asumir que el asistente intercambiaba dos objectos en vez de que se fue al estante equivocado. Por ejemplo, si debe de estar 2 objectos en la estante de bebidas y ningúno en comida, el algoritmo va a eligir un diagnostico que reporte 2 objectos en bebidas sobre uno que reporte 1 en cada uno. Primero, para cada objecto fuera del lugar genera una lista de posible estantes donde podría estar el objecto sin contradicta las observaciones. Luego, eligir el primer objecto y buscar para otro objecto que podría intercambiar con este. Por ejemplo, si creen que la cerveza está en estante 1 pero despues de las observaciones se da cuenta que solamente puede estar en estante 2 o estante 3, busca a cualquier objecto también fuera de su lugar que está en estante 2 y puede ser en estante 1, o que está en estante 3 pero puede ser en estante 1. Si no hay ningún objecto que puede intercambiar con esto, elige uno de los estantes y procede al próximo objecto. En este manera se puede reducir el espacio de búsqueda por eliminar escenarios que no van a generar un diagnostico que resultará en una cuenta parecido al esperado.
\subsection*{Ejemplos de uso}

\subsubsection{}
	\begin{tabular}{| l | l | l | l |}
		\hline
						& Shelf 1 & Shelf 2 & Shelf 3 \\ \hline
		Creencias		& Refresco, Creveza & Sopa & Galletas \\ \hline
		Observaciones	& Refresco, Creveza & & \\
		\hline
	\end{tabular}
	\begin{lstlisting}
?- open_kb('KB_Original.txt', KB), diagnosis(KB, Creencias_New, Acciones).
Creencias_New = [refresco=>shelf1,cerveza=>shelf1,sopa=>shelf2,galletas=>shelf3],
Acciones = [mover(l0,shelf1),colocar(refresco),colocar(cerveza),mover(shelf1,shelf2),colocar(sopa),mover(shelf2,shelf3),colocar(galletas)],
	\end{lstlisting}
	\textit{Apuntes: No hubo contradiciones, entonces las creencias quedan sin cambios} \\

\subsubsection{}
	\begin{tabular}{| l | l | l | l |}
		\hline
						& Shelf 1 & Shelf 2 & Shelf 3 \\ \hline
		Creencias		& Refresco, Creveza & Sopa & Galletas \\ \hline
		Observaciones	& Refresco, Creveza, Sopa & & \\
		\hline
	\end{tabular}
	\begin{lstlisting}
?- open_kb('KB_Original.txt', KB), diagnosis(KB, Creencias_New, Acciones).
Creencias_New = [refresco=>shelf1,cerveza=>shelf1,galletas=>shelf3,sopa=>shelf1],
Acciones = [mover(l0,shelf1),colocar(refresco),colocar(cerveza),colocar(sopa),mover(shelf1,shelf3),colocar(galletas)],
	\end{lstlisting}
	\textit{Apuntes: No hay nada que contradiga que las galletas quedan en estante 3}

\section{Módulo de Toma de Decisión}

\paragraph{Descripción} El módulo de toma de decisión toma la creencia nueva generado por el módulo de diagnostico y genera una decisión sobre qué hacer para cumplir con el objectivo.

\subsection*{Nombre}
	\begin{verbatim}
		decision(KB, Diagnostico, Decision)
	\end{verbatim}

\subsection*{Argumentos}
	\textit{KB}: (Input) Base de conocimiento a consultar \\
	\textit{Diagnostico}: (Input) Lista de creencias representado en forma de clave y valor \\
	\textit{Decision}: Lista de decisiones de metas de cumplir 

\paragraph{Búsqueda} 

\subsection*{Ejemplos de uso}

En los siguientes ejemplos, el cliente pide un refresco.

\subsubsection{}
	\begin{tabular}{| l | l | l | l |}
		\hline
						& Shelf 1 & Shelf 2 & Shelf 3 \\ \hline
		Creencias		& Refresco, Creveza & Sopa & Galletas \\ \hline
		Diagnóstico		& Refresco, Creveza & & \\
		\hline
	\end{tabular}
	\begin{lstlisting}
?- open_kb('KB_Original.txt', KB), 
	\end{lstlisting}
	\textit{Apuntes: No hay nada que reacomodar}

\section{Módulo de Planeación}

\paragraph{Descripción} El módulo de planeación toma la decisión emitido por el módulo de toma de decición y planea los acciones necesarios para cumplir con los objectivos.

\subsection*{Nombre}
	\begin{verbatim}
		plan(KB, Diagnostico, Decision, Actions)
	\end{verbatim}

\subsection*{Argumentos}
	\textit{KB}: (Input) Base de conocimiento a consultar \\
	\textit{Diagnostico}: (Input) Lista de creencias representado en forma de clave y valor \\
	\textit{Decision}: (Input) Lista de decisiones de metas de cumplir 
	\textit{Actions}: Lista de decisiones de metas de cumplir 

\paragraph{Búsqueda} 

\subsection*{Ejemplos de uso}

\end{document}
